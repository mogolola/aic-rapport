%% Copyright (C) 2008 Johan Oudinet <oudinet@lri.fr>
%%  
%% Permission is granted to make and distribute verbatim copies of
%% this manual provided the copyright notice and this permission notice
%% are preserved on all copies.
%%  
%% Permission is granted to process this file through TeX and print the
%% results, provided the printed document carries copying permission
%% notice identical to this one except for the removal of this paragraph
%% (this paragraph not being relevant to the printed manual).
%%  
%% Permission is granted to copy and distribute modified versions of this
%% manual under the conditions for verbatim copying, provided that the
%% entire resulting derived work is distributed under the terms of a 
%% permission notice identical to this one.
%%  
%% Permission is granted to copy and distribute translations of this manual
%% into another language, under the above conditions for modified versions,
%% except that this permission notice may be stated in a translation
%% approved by the Free Software Foundation
%%  
\chapter{Background}
\label{sec:chapter2}

\section{Hurricane Trajectory Forecasting}
\label{sec:chapter2_1}
Cyclones, hurricanes or typhoons are words for the same phenomena: a rare and complex event characterized by strong winds surrounding a low-pressure area. Hurricane is one of the most severe natural disasters that cost tremendous damage and death each year. During its life cycle, a hurricane can expend as much as 10000 nuclear bombs \cite{hurricane_energy}. Since 2000, over 45000 people were killed by hurricanes. In 2005, hurricane Katrina had cost as much as 125 billion US Dollars worth of damage. The explosion of the population also raises the risk of potential damage made by hurricanes\cite{peduzzi2012global}. In 2010, it was estimated that 1.53 billion people lived in hurricanes prone areas in 81 different countries and territories. 133.7 millions of people were exposed to hurricanes. The number has increased three times compared to the year 1970. Hurricane's huge threat to human society makes the forecast of hurricane trajectory crucial, in order for people living in that area to be alarmed and get some time to protect their goods and evacuate. Improving the prediction of hurricane movements has a significant impact on society.



\begin{table}[b]
	\centering
	\begin{tabular}{ccc}
		\rowcolor[HTML]{EFEFEF} 
		\multicolumn{3}{c}{\cellcolor[HTML]{EFEFEF}Saffir-Simpson Hurricane Scale}                                                                                \\
		\rowcolor[HTML]{00D2CB} 
		\cellcolor[HTML]{00D2CB}{\color[HTML]{333333} }                           & \multicolumn{2}{c}{\cellcolor[HTML]{00D2CB}{\color[HTML]{333333} Wind Speed}} \\
		\rowcolor[HTML]{00D2CB} 
		\multirow{-2}{*}{\cellcolor[HTML]{00D2CB}{\color[HTML]{333333} Category}} & {\color[HTML]{333333} mph}           & {\color[HTML]{333333} knots}           \\
		\rowcolor[HTML]{FE0000} 
		5                                                                         & \textgreater{}=156                   & \textgreater{}=135                     \\
		\rowcolor[HTML]{F8A102} 
		4                                                                         & 131-155                              & 114-134                                \\
		\rowcolor[HTML]{FFCC67} 
		3                                                                         & 111-130                              & 96-113                                 \\
		\rowcolor[HTML]{FFFE65} 
		2                                                                         & 96-110                               & 84-95                                  \\
		\rowcolor[HTML]{FFFC9E} 
		1                                                                         & 74-95                                & 65-83                                  \\
		\rowcolor[HTML]{EFEFEF} 
		\multicolumn{3}{c}{\cellcolor[HTML]{EFEFEF}Non-Hurricane Scale}                                                                                           \\
		\rowcolor[HTML]{9AFF99} 
		\begin{tabular}[c]{@{}c@{}}Tropical\\ Storm\end{tabular}                  & 39-73                                & 34-64                                  \\
		\rowcolor[HTML]{34FF34} 
		\begin{tabular}[c]{@{}c@{}}Tropical \\ Depression\end{tabular}            & 0-38                                 & 0-33                                  
	\end{tabular}
	\caption{The Saffir-Simpson Hurricane wind scale and Non-Hurricane Scale}
	\label{fig::Saffir_Simpson}
\end{table}

\subsection{Formation of Hurricanes}
The evolution and path of hurricanes depend on many factors at different scales and altitudes. All hurricanes are formed over warm ocean water near the equator. Warm and moist air over the ocean rises upward, which cause low-pressure area below. The air from surrounding high-pressure areas flows over into low-pressure areas, and new air is heated and rise upward again. As warm and moist air rise and cools off, the water in the air forms clouds. Figure \ref{hurricane_section} shows the wind flows inside storm. The storm grows up and rotates as the cycle continues to be fed by ocean's heat and water evaporating from the surface. When the winds in the rotating storm reach 39mph(miles per hour, equivalent to 63km/h), the storm is called a "tropical storm". When the wind speeds reach 74 mph, the storm is officially called a "tropical cyclone", or hurricane. The Saffir–Simpson hurricane scale (SSHS) classifies hurricanes into five categories distinguished by the intensities of their sustained winds. The highest classification in the scale, category 5, consists of storms with sustained wind speeds higher than 156mph, See Figure\ref{fig::Saffir_Simpson}. To avoid ambiguity, the term 'storm' will be used for all kinds of categories in SSHS, and the term 'hurricane' will only be used for those in category 5 in the report. \\
A hurricane can be very unpredictable. The hurricane movement can perform loops, hairpin turns, and sharp curves. Its intensity can also change dramatically in a very short time. In 2005, hurricane Katrina was only classified as a tropical storm when it made its first landfall in Florida. After emerged into the Gulf of Mexico, it rapidly strengthened into a Category 5 hurricane in two days. Its second landfall finally caused the largest damage in history. The track and evolution of Hurricane Katrina are shown in Figure \ref{fig:Katrina}  Global warming is also considered to influence hurricane activities. It is estimated that global warming will cause more intense hurricanes globally in the coming century and have higher rainfall rates\cite{knutson2013dynamical}, which may lead to both more representative and more consistent error statistics for forecasting.

\begin{figure}[t]
	\begin{center}
		\epsfxsize=0.75\hsize \epsfbox{figs/hurricane_section.png}
	\end{center}
	\caption{Graphic of wind flow inside storm \cite{hurricane_energy}}
	\label{fig:hurricane_section}
\end{figure}



\section{Forecast Method for Meteorologists}
Tracks and intensity are the two main goals of the forecast. Today, the forecasts (track and intensity) are provided by a numerous number of guidance models \cite{nhc_models}. In the 70's and 80's the models were statistical models based on historical relationships between storm behavior and various other parameters\cite{demaria2005further}. Today statistical models are mostly used for testing and comparing new models. With the increase of computing capacity and data assimilation techniques, dynamic models gradually replaced statistical models. Dynamical models solve the physical equations governing the motions in the atmosphere, but they are very complex and computationally demanding. These models usually run on high-speed supercomputers. 

Guidance models are characterized as either early or late, depending on whether or not they are available to the Hurricane Specialist during the forecast cycle. \cite{nhc_models} The late model can take hours to run. For example, if the NWS/Global Forecast System (GFS) runs at 12UTC, the result will not be available until 16UTC. The Dynamic models, in general, are late models. Although they can have a very precise forecast, the bottleneck is the computing speed. If the computation lasts for days, it doesn't make sense to make a prediction since the prediction will already be outdated. But fortunately, there is a technique that allows to take the latest available run of a late model and adjust its forecast to apply to the current synoptic time and initial conditions\cite{nhc_models}. Current national forecasts are typically driven by consensus methods able to combine different dynamical models. The forecasts are usually calculated to have from 12h-predictions, 24h-predictions, 36h-predictions and up to 120h-predictions. 

\begin{figure}[t]
	\begin{center}
		\epsfxsize=0.75\hsize \epsfbox{figs/Katrina.png}
	\end{center}
	\caption{Hurricane Katrina track. Uses the color scheme from the Saffir-Simpson Hurricane Scale. The points show the location of each storm at six-hour intervals. The color represents the storm's maximum sustained wind speeds as classified in the Saffir-Simpson Hurricane Scale (see \ref{fig::Saffir_Simpson}), and the shape of the data points represent the nature of the storm \cite{katrina_track}}
	\label{fig:Katrina}
\end{figure}


\section{Machine Learning in Climate Science}

The hurricane statistical forecasting models perform poorly with respect to the dynamical models, even if the database made of past hurricane is constantly growing. The machine learning methods, able for example to capture non-linearity and complex relations, have only been scarcely tested. Artificial Neural Network have recently shown their efficacy in a various number of forecasting tasks. Particularly, convolutional neural networks (CNN) have raised attention as they are suited for large imaging data. A convolutional LSTM model has been used for precipitation forecast in a promising study \cite{xingjian2015convolutional}. Another recent study predicts the evolution of sea surface temperature maps by combining CNN with physical knowledge \cite{de2017deep}. CNNs have also been used for the detection of extreme weather like hurricanes from meteorological variables patches\cite{racah2017extremeweather}. These studies show the strong potential on various climate problems. 

To our knowledge, only two preliminary studies have tackled the hurricane forecast tracking using machine learning: the first one uses random forests on local reanalysis histograms \cite{liberge2011prevision}, however, it was tested only on tropical storms in 2015 in North Atlantic, and the mean error of 6h-forecasts seem to indicate poor results (more than 60km). The second uses a sparse recurrent neural network from trajectory data \cite{moradi2016sparse}. It was tested on only four hurricanes and also seem to give large distance errors (mean 6h-forecast error is 72km). 


\chapter{Prelimiaries}

\section{Problem Setting}
The goal of the research is to predict the trajectory of Hurricane using the information from the past storms since 1979. The task is shown in Figure \ref{fig::task}. We aim at building an end-to-end model using two types of data (reanalysis and hand-crafted features), for each time step of each storm, we want to independently predict its displacement in the future(depending on forecast cycle). 

\begin{figure}[H]
	\begin{center}
		\epsfxsize=0.75\hsize \epsfbox{figs/storm_shema.png}
	\end{center}
	\caption{General architecture: the two types of data are feeding two neural networks trained separately. The final fused network is re-trained before predicting the forecast track. }
	\label{fig::task}
\end{figure}

The objective of the internship is to study for using deep learning as an alternative methodology for hurricane trajectory forecasting. Our research is based on a large body of work on recent advances in convolutional neural networks. We will briefly describe the related literature to our approach in the following part of this chapter.  

\section{Neural Network for sequence modeling}
From the machine learning perspective, hurricane track forecasting can be essentially taken as a spatiotemporal sequence forecasting problem, where the input can be spacial atmospheric fields evolving with time, and the output can be a fixed number (1 or more) of hurricane's future displacement. The problem is difficult to solve in the first place due to the high dimensionality of the spatiotemporal sequences. It is important to build a prediction model that could effectively learn patterns from both its spatial and temporal structure. 

Recent advances in deep learning have shown promising results on enormous pattern recognition tasks, such as ImageNet Large Scale Visual Recognition Challenge (ILSVRC) \cite{russakovsky2015imagenet} \cite{krizhevsky2012imagenet}  \cite{szegedy2015going} and natural language processing \cite{goldberg2014word2vec} \cite{sutskever2014sequence}. Weather deep learning could open a new track for hurricane track forecasting is a key problem we want to tackle in this research. The artificial neural network has also been largely used for sequence modeling. In general, Convolutional Neural Network (CNN) and Recurrent Neural Network(RNN) are the two main architectures that are dedicated to sequence modeling. 

For most deep learning practitioners, Recurrent Neural Network(RNN) is the default choice for sequence modeling. RNN is a type of neural network that maintains a hidden vector backpropagated through time. Basic RNN is known difficult to train. For long sequences, backpropagation causes gradient exploding or vanishing problem. More advanced architectures are used instead in practice such as LSTM\cite{hochreiter1997long} and GRU \cite{collobert2008unified}, which are proven to be robust for having long-range dependencies, and doesn't suffer from gradient exploding or vanishing. The family of RNN has been widely used in language modeling\cite{graves2013generating} and machine translation\cite{sutskever2014sequence}.  While for our problem, standard LSTM or GRU are not suitable models for end-to-end learning because they cannot tackle with high-dimensional input tensor data. 

CNN has been applied to sequences for a long time. In the earlier days Time Delay Neural Network (TDNN) was used for speech recognition \cite{waibel1990phoneme} that inspired largely to CNN. Later CNN has been widely applied to NLP tasks and achieved excellent results in sentence modeling \cite{kalchbrenner2014convolutional}, classification \cite{kim2014convolutional}, prediction \cite{collobert2008unified}... Compared to RNNs, CNNs are faster and use less memory bandwidth on the same sequence modeling tasks. In particular, a recent application of CNN to machine translation \cite{gehring2017convolutional} achieved state-of-art accuracy at nine times the speed of recurrent neural systems. Another study has carried out a comprehensive comparison between a generic convolutional model and canonical recurrent models such as LSTMs and GRUs on typical sequence modeling tasks that are commonly used to benchmark RNN themselves, and concluded "a simple convolutional architecture outperforms canonical recurrent networks across a diverse range of tasks and datasets, while demonstrating longer effective memory" \cite{bai2018empirical}. 

There are also works that aim at combining RNN and CNN architectures. Convolutional LSTM replaces fully-connected layers in LSTM with convolutional layers to capture the spatiotemporal structure of the data in a weather nowcasting problem \cite{xingjian2015convolutional}. This work shows promise in combining different architectures of neural networks. But it needs sufficient training data and is difficult to train. In our research, we consider using the convolutional network as our starting point.


\section{Convolutional Neural Network}
 An example of CNN is presented in Figure \ref{fig:LeNet} It is usually made of several convolutional layers followed by some fully connected layers. The convolutional layer is inspired by the animal's cortex visual system, where each neuron only processes data for its receptive field. Between two successive convolutional layers, there is usually also sub-sampling layers also known as pooling layers. The Advantage of CNN is that it learns both spatial and temporal differences in different scales and thus could extract features automatically without learning a massive number of parameters. The CNNs are widely adopted as a very effective model for analyzing images or images-like data for pattern recognition. 

Modern CNNs tend to be deep with a large number of hidden layers. AlexNet is seen as a breakthrough when it won the 2012 ImageNet LSVRC-2012 competition by a large margin (reached 15.3\% compared to the previous best 26.2\% in error rates on top-5 classification problem). AlexNet has five convolutional layers followed by three fully-connected layers. Researchers continually introduce new architectures of CNNs that are deeper, more complex and have better precision on ImageNet. The CNNs, as deep learning in general, is an empirical construction of a learning algorithm. Some works in CNNs show that CNNs should have more layers for the hierarchical representation of visual data to work\cite{he2016deep}. VGG Net\cite{simonyan2014very} is one of the most influential works that gives a guideline to design a CNN architecture. A VGG Net applies alternatively convolutional layers and max-pooling layers through the whole network and put several fully connected layers at the end of the network to generate output. Rather than using relatively large receptive fields in the first convolutional layers (e.g., 11x11 with stride 4 in AlexNet\cite{krizhevsky2012imagenet}), the author suggests using very small 3x3 receptive fields throughout the whole network. The reason is that there could be more non-linear layers and a smaller number of parameters, and effectively to have scopes from small to large receptive fields. It turns out that the more important for VGG Net becomes tunning the depth of the network without worrying about other hyper-parameters like kernel size. It is the simplicity and effectiveness of VGG Net that makes it among the most popular CNN architectures.

CNN is known to be tedious to train. To build an excellent CNN, researchers have proposed different methods. Some of them are crucial to successfully train a CNN. Training a CNN with a huge number of learnable parameters usually require a considerable amount of examples. The data Augmentation methods such as flipping or randomly cropping can effectively increase the network's performance when there is only a limited number of examples. Proper input data normalization and weights initialization are essential steps to accelerate convergence during the training. To realize it, the input data can be zero-centered and normalized at each channel to have a similar distribution. Weights are usually initialized by random weights drawn by Gaussian distributions\cite{krizhevsky2012imagenet}, but it can lead to a very slow convergence when training very deep CNN. A paper\cite{glorot2010understanding} shows that the reason is that the norm of the outputs in each layer does not initially sum to 1. If the model has too many layers, the norm will be exploded. The author proposes a method called 'Xavier' initialization to ensure the outputs in each layer are initially standard normally distributed. 'Xavier' initialization is based on the assumption that activations are linear, which is not valid on rectifier nonlinearities(ReLU). Another paper\cite{he2015delving} suggests a more robust initialization that deals with the rectifier nonlinearities. Batch Normalization(BN) layers are also recommended to be used to maintain the distribution of the outputs in each layer during training and thus to accelerate convergence\cite{krizhevsky2012imagenet}. 


\begin{figure}
	\begin{center}
		\epsfxsize=0.75\hsize \epsfbox{figs/tLKYz.png}
	\end{center}
	\caption{Architecture of LeNet, shows a series of layers in CNN that learn hierarchical features representation \cite{lecun1998gradient} }
	\label{fig:LeNet}
\end{figure}




%%% Local Variables: 
%%% mode: latex
%%% TeX-master: "rapportM2R"
%%% End: 
