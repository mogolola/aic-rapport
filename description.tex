%% Copyright (C) 2008 Johan Oudinet <oudinet@lri.fr>
%%  
%% Permission is granted to make and distribute verbatim copies of
%% this manual provided the copyright notice and this permission notice
%% are preserved on all copies.
%%  
%% Permission is granted to process this file through TeX and print the
%% results, provided the printed document carries copying permission
%% notice identical to this one except for the removal of this paragraph
%% (this paragraph not being relevant to the printed manual).
%%  
%% Permission is granted to copy and distribute modified versions of this
%% manual under the conditions for verbatim copying, provided that the
%% entire resulting derived work is distributed under the terms of a 
%% permission notice identical to this one.
%%  
%% Permission is granted to copy and distribute translations of this manual
%% into another language, under the above conditions for modified versions,
%% except that this permission notice may be stated in a translation
%% approved by the Free Software Foundation
%%  
\chapter{Background}
\label{sec:chapter2}

\section{Hurricane Trajectory Forecasting}
\label{sec:chapter2_1}
Cyclones, hurricanes or typhoons are words for the same phenomena: a rare and complex event characterized by strong winds surrounding a low pressure area. Hurricanes is one of the most severe natural disaster that cost tremendous damage and death in each year. An average hurricane can release as much energy in a day as explosion of half a million small atomic bombs. Since 2000, over 45000 people were killed by hurricanes. In 2005, hurricane Katrina has cost as much as 125 billions US Dollars worth of damage. Explosion of population also raise the risk of potential damage made by hurricanes.\cite{peduzzi2012global}. In 2010, It is estimated that 1.53 billion people live in hurricanes prone areas in 81 different countries and territories. 133.7 millions of people is exposed to hurricanes, the number has increased three times compared to the year 1970. That makes the forecast of hurricane trajectory so important, if we know when and where the hurricanes is going to make landfall, people live in that area will be alarmed and get some time to protect their goods and evacuate. Improving the prediction of hurricane movements have enormous significance on society.

\subsection{Formation of Hurricanes}
The evolution and path of hurricanes depends on many factors at different scales and altitudes. All hurricanes are formed over warm ocean water near the equator. Warm and moist air over the ocean rises upward, which cause low pressure area below. Air from surrounding areas with higher pressure flow over into low pressure area, and new air are heated and rise upward again. As warm and moist air rise and cools off, the water in air forms clouds. The storm grows up and rotate as the cycle continue fed by ocean's heat and water evaporating from the surface. When the winds in the rotating storm reach 39mph, the storm is called a "tropical storm". When the wind speeds reach 74 mph, the storm is officially called a "tropical cyclone", or hurricane. The Saffir–Simpson hurricane scale (SSHS), classifies hurricanes into 5 categories distinguished by the intensities of their sustained winds. The highest classification in the scale, category 5, consists of storms with sustained wind speeds higher than 156mph.\\
Hurricane can be very unpredictable. The movement of hurricane sometimes performs loops, hairpin turns, and sharp curves. Its intensity can also change violently in a very short time. In 2005, After hurricane Katrina made its first landfall in Florida it was only classified as tropical storm. After emerged into Gulf of Mexico, it rapidly strengthened into a Category 5 hurricane and its second landfall finally caused the largest damage from natural disaster in history. Global warming is also considered to influence hurricane activities. it is estimated that global warming will cause hurricanes in the coming century to be more intense globally and have higher rainfall rates than present-day hurricanes \cite{knutson2013dynamical}, which may lead to both more representative and more consistent error statistics for forecasting.


\subsection{Forecast Method for Meteorologists}
Tracks and intensity are the two main goal of the forecast. Today, the forecasts (track and intensity) are provided by a numerous number of guidance models \cite{nhc_models}. The original best model was statistical model based on historical relationships between storm behavior and various other parameters. It was the major forecasting model until 1980's \cite{demaria2005further}. Today it is mostly used for testing and comparing new models. With the increase of computing capacity and data assimilation techniques, Dynamic model gradually replace statistical model. Dynamical models solve the physical equations governing motions in the atmosphere, they are very complex and computationally demanding. These models normally runs on high-speed super computers. The bottle neck is the computing speed. If the computation last for days, it wouldn't make sense to make prediction since the prediction will already be out of date. The statistical model, in contrast, does not demand high computational resources, but are less accurate than dynamic models. Current national forecasts are typically driven by consensus methods able to combine different dynamical models.

\subsection{Limitation of current Forecasting method}
Guidance models are characterized as either early or late, depending on whether or not they are available to the Hurricane Specialist during the forecast cycle. \cite{nhc_models} The late model can take hours to run. For example, if the NWS/Global Forecast System (GFS) runs at 12UTC, the result will not be available until 16UTC. The Dynamic models, in general, are late models. Although they can have very precise forecast, they won't be able to release at once. The current solution would be that use the latest available run of their models to release a first forecast. Then the forecast is adjusted and shifted when there are more available. The National Hurricane Center of NOAA produce forecast from 6h to 126h. Out of some reasons, they don't provide all forecast throughout a hurricane life cycle, and there is almost no forecast under 24h hours forecast (\textbf{to be verified})

\subsection{Reanalysis}
\textbf{to be verified whether it is necessary to add this section }



\section{CNN in climate science}
The hurricane statistical forecasting models perform poorly with respect to dynamical models, even if the database made of past hurricane is constantly growing. The machine learning methods, able for example to capture non-linearity and complex relations, have only been scarcely tested. However, they have recently shown their efficacy in a various number of forecasting tasks. Particularly, convolutional neural networks (CNNs) have raised attention as they are suited for large imaging data.A convolutional LSTM model has been used for precipitation forecast in a promising study \cite{xingjian2015convolutional}. Another recent study predicts the evolution of sea surface temperature maps by combining CNNs with physical knowledge \cite{de2017deep}. The CNNs have also been used for the detection of extreme weather like hurricanes from meteorological variables patches. \cite{racah2017extremeweather}.

To our knowledge, only two preliminary studies have tackled the hurricane forecast tracking using machine learning: the first one uses random forests on local reanalysis histograms \cite{liberge2011prevision}, however the mean error of 6h-forecasts seem to indicate poor results (more than 60km). The second uses a sparse recurrent neural network from trajectory data \cite{moradi2016sparse}, but it was tested on only 4 hurricanes and seem aslo to give large distance errors (mean 6h-forecast error is 72km). Their results seems not impressive compared to the existing forecasting method. In this study, we expect to explore Deep Convolutional Neural Network to improve hurricane trajectory forecasting.



%%% Local Variables: 
%%% mode: latex
%%% TeX-master: "rapportM2R"
%%% End: 
