%% Copyright (C) 2008 Johan Oudinet <oudinet@lri.fr>
%%  
%% Permission is granted to make and distribute verbatim copies of
%% this manual provided the copyright notice and this permission notice
%% are preserved on all copies.
%%  
%% Permission is granted to process this file through TeX and print the
%% results, provided the printed document carries copying permission
%% notice identical to this one except for the removal of this paragraph
%% (this paragraph not being relevant to the printed manual).
%%  
%% Permission is granted to copy and distribute modified versions of this
%% manual under the conditions for verbatim copying, provided that the
%% entire resulting derived work is distributed under the terms of a 
%% permission notice identical to this one.
%%  
%% Permission is granted to copy and distribute translations of this manual
%% into another language, under the above conditions for modified versions,
%% except that this permission notice may be stated in a translation
%% approved by the Free Software Foundation
%%  
\chapter{Introduction}
\label{sec:intro}
The research is carried out in a mixed team with members from two laboratories: Laboratoire de l'Accelerateur Lineaire and Inria Saclay. The team leaders combine Monteleoni’s expertise in climate informatics and algorithms for learning in the presence of spatial and temporal non-stationarity, with Charpiat’s expertise in deep learning architectures and applications. This chapter outline the related organizations, teams and the internship subject.

\section{Laboratory of the Linear Accelerator}
\label{sec:laboratory}
The Laboratory of the Linear Accelerator (LAL) is under the joint supervision of the Universite Paris-Sud and the Institut National de Physique Nucleaire et de Physique des Particules (IN2P3) of the CNRS. There are 309 agents on January 1st, 2017, including 135 researchers and 174 engineers and technicians. As the name suggested, the research activity of the LAL is centered on particles physics, supplemented by a strong component in cosmology and astrophysics. Most of the researchers have physicist background. Recently, LAL is enforcing its connection with nearby CNRS institutes and University Paris-Saclay to establish a coherent unit and share competencies and resources. 

\section{Inria research center Saclay}
The French Institute for Research in Computer Science and Automation (Inria) is a French national research institution focusing on computer science and applied mathematics. Created in 1967, Inria now has 8 research centers located throughout France, over 2400 employees from 102 countries. 184 project-teams lead research in Inria and 4400 scientific publications were published per year (statistics in 2017)\cite{inria_statistics}. Inria promotes “scientific excellence for technology transfer and society”. The duty of Inria is to respond to the challenges of digital transformation in multidisciplinary applications in fields as diverse as health, energy, security and privacy protection, environment, climate, transportation, economy, finance, agriculture...

The Inria Saclay-Île-de-France Research Centre was created in 2008 and is located at the heart of the main national research and higher education cluster. It is also a member of the Université Paris Saclay, and a major actor in the French Investments for the Future Programme (Idex, LabEx, IRT, Equipex). 450 researchers and engineers from Inria and its partners work in the center's 31 teams. The center is particularly active in three major areas: data and knowledge; safety, security and reliability; modeling, simulation and optimization (with priority given to energy)\cite{inria_overview}.

\section{Related teams}
\label{teams}
Applied Statistics and Machine Learning (AppStat) is a group inside LAL with an initiative focusing on interdisciplinary projects between physics and computer science. Now the group acts in a variety of domains including image and music processing, bioinformatics, climate informatics, software engineering, grid control, experimental physics... The head of AppStat is Balazs Kegl. Members of the groups come from physics or computer science background. Each person carried on his or her research, but they can exchange ideas and share the experience with other research intuitions. Balazs Kegl and the team are also acting as coordinators of Paris-Saclay Center for Data Science (PSCDS), which forms a larger machine learning community. PSCDS allow researchers in the frame of Paris Saclay to create data challenge from their research and let others compete and collaborate. The goal is to link the people in the different background (both data scientists and domain experts) to cooperate in a complementary way\cite{balazs-kegl}.  

TAO is the mixed INRIA Saclay - CNRS - LRI, University Paris-Sud research group interested in the interplay of Machine Learning (A for Apprentissage) and Optimization (O). Since 2016 team TAO in Inria is stopped and is replaced by two new teams: RANDOPT(RANDomized OPTimization) and TAU(TAckling the Underspecified). RANDOPT team locates at the Applied Maths Center at Ecole Polytechnique, works on all types of blackbox optimization especially CMA-ES type. TAU team is situated at LRI(laboratoire de recherche en informatique) in University Paris-Sud, aims to tackle the vagueness of the Big Data purposes\cite{TAO} in three respects: providing machine with \textbf{commen sense}, steering a Big Data system in a dynamic environment, and trading-off between efficiency and protecting individual freedom and privacy\cite{TAU}. The main research dimensions in TAU involve Causal Modeling (required to support prescriptive Big Data), Deep Learning (related to constructive representations, and their compositionality), Optimization and Meta-Optimization (including sequential decision making and categorization of problems), and Big-Data Driven Design.


\section{Research team regarded to internship subject}
The research is carried out in a mixed team with members from the two groups above. Claire Monteleoni is a visiting professor to AppStat in LAL, she works on machine learning algorithms and recently focuses on climate informatics. Guillaume Charpiat is a deep learning researcher in TAO/TAU team in the INRIA Saclay. His research interest covers neural network theory, computer vision, and optimization. Balazs Kegl is a researcher in the field of computer science in LAL and head of AppStat. His has a broad research interest in machine learning while largely focusing on standardizing workflow for machine learning production and facilitating the use of machine learning methods for the non-expert. Sophie Giffard-Roisin is a post-doctoral researcher at AppStat team in LAL. Her research focuses on applied maths and computer science, and more particularly machine learning techniques for computational sustainability and climate informatics.  


\section{Internship Subject}
\label{sec:subject}
In recent years, a sequence of Atlantic hurricanes surprised and crippled several highly inhabited areas. Cuba had very little warning of the updated track of hurricane Irma before it
was hit\cite{NYT}, and Puerto Rico was just reeling from hurricane Irma when it was devastated by hurricane Maria two weeks later. Better understanding and predicting the development and movement of such severe storms (hurricanes, cyclones, typhoons: names differ by region) is critical to protecting communities and ecosystems. This research focuses on exploring the capacity of machine learning approaches to improve predictions of severe storms. The study addresses the following questions:


\begin{enumerate}[label=\zdyxh*., leftmargin=5em]
	\item  Can machine learning contribute to understanding and predicting the tracks and intensity of severe storms (hurricanes, cyclones, typhoons)? 
	\item is deep learning an effective approach to such problems?
\end{enumerate}




\section{Outlines of the rest of the report}
The rest of the report is organized as follows:

Chapter 2 describes the context of hurricane track forecasting. Then Chapter 3 sets up the problem and presents the related prior knowledge to the research. Our proposed method is described in Chapter 4. Chapter 5 will be dedicated for the experiments and analysis. Chapter 6, as a concluding chapter, summarizes the main findings and identified future research opportunities as well as critiques of the current research.


%%% Local Variables: 
%%% mode: latex
%%% TeX-master: "rapportM2R"
%%% End: 
