%% Copyright (C) 2008 Johan Oudinet <oudinet@lri.fr>
%%  
%% Permission is granted to make and distribute verbatim copies of
%% this manual provided the copyright notice and this permission notice
%% are preserved on all copies.
%%  
%% Permission is granted to process this file through TeX and print the
%% results, provided the printed document carries copying permission
%% notice identical to this one except for the removal of this paragraph
%% (this paragraph not being relevant to the printed manual).
%%  
%% Permission is granted to copy and distribute modified versions of this
%% manual under the conditions for verbatim copying, provided that the
%% entire resulting derived work is distributed under the terms of a 
%% permission notice identical to this one.
%%  
%% Permission is granted to copy and distribute translations of this manual
%% into another language, under the above conditions for modified versions,
%% except that this permission notice may be stated in a translation
%% approved by the Free Software Foundation
%%  
\chapter{Conclusion and Discussion}

We showed a promising deep learning framework for storm track forecasting. We demonstrated the benefit of coupling two types of data (past tracks and wind fields) in an efficient fusion model. Our results on a large database (90 000 time steps from 3000 storms) from different oceanic basins are showing 6h predictions with less than 33km error. Moreover, the error on only hurricane data points (without depressions) drops to 25.8km. We think that the use of such deep learning methods can help the current forecast modellers by providing a complementary prediction that could be integrated in some consensus methods.

%%% Local Variables: 
%%% mode: latex
%%% TeX-master: "rapportM2R"
%%% End: 
